\documentclass[letterpaper,11pt]{article}

\usepackage{amsmath,amssymb}
\usepackage[utf8]{inputenc}

\usepackage[left=1in,right=1in,top=1in,bottom=1in]{geometry}

\usepackage{hyperref}
\usepackage{url}

% generally i use verbose
\usepackage[sorting=none,backend=bibtex,style=ieee,maxbibnames=99,maxcitenames=99]{biblatex}
\addbibresource{references_consistent_July2020.bib}
% consistent uses Daniel J Tward everywhere

\newcommand{\comment}[1]{}

\usepackage[T1]{fontenc}
\usepackage[sc,osf]{mathpazo}

\def\name{Daniel Jacob Tward, PhD}

\hypersetup{
  colorlinks = true,
  linkcolor = black,
  urlcolor = black,
  pdfauthor = {\name},
  pdfkeywords = {biomedical engineering, neuroimaging, shape analysis},
  pdftitle = {\name: Curriculum Vitae},
  pdfsubject = {Curriculum Vitae},
  pdfpagemode = UseNone
}

% Customize page headers
\usepackage{fancyhdr}

\usepackage{lastpage}
%\pagestyle{myheadings}
\pagestyle{fancy}

\cfoot{\thepage { of } \pageref{LastPage}}

\renewcommand{\headrulewidth}{0pt}
%\lhead{\name} % put it on second page
\lhead{}


% Custom section fonts
\usepackage{sectsty}
\sectionfont{\rmfamily\mdseries\Large}
\subsectionfont{\rmfamily\mdseries\itshape\large}

% Other possible font commands include:
% \ttfamily for teletype,
% \sffamily for sans serif,
% \bfseries for bold,
% \scshape for small caps,
% \normalsize, \large, \Large, \LARGE sizes.

% Don't indent paragraphs.
\setlength\parindent{0em}


\usepackage{enumitem}
\newlength{\dateindent}
\setlength{\dateindent}{0.8in}
%\setlist[itemize]{leftmargin=\dateindent}



\begin{document}
% Place name at left
{\huge \name}

% Alternatively, print name centered and bold:
%\centerline{\huge \bf \name}

%\vspace{0.25in}


  \href{https://www.ucla.edu//}{University of California Los Angeles}, 
  \href{https://compmed.ucla.edu/}{Departments of Computational Medicine} and Neurology\\
 %\href{http://www.bmap.ucla.edu/}{Ahmanson-Lovelace Brain Mapping Center}
Email: \href{mailto:dtward@mednet.ucla.edu}{\tt dtward@mednet.ucla.edu} , Homepage: \href{http://danieltward.com}{\tt danieltward.com}
%    Homepage: & \href{http://www.stat-or.unc.edu/}{\tt http://www.stat-or.unc.edu/} \\


\newcommand{\dancol}[2]{%
\begin{minipage}{5.9in}%
#1 %
\end{minipage}~%
\begin{minipage}{0.6in}%
\hfill%
#2%
\end{minipage}%
\vspace{1em}%
}


\vspace{-1em}
\section*{Education}
\vspace{-0.75em}

\dancol{\textbf{The Johns Hopkins University School of Medicine}, Baltimore, MD\\
PhD in Biomedical Engineering}{2017}

\dancol{\textbf{University of Toronto}, Toronto, ON, Canada\\HBSc in physics and physiology (dual specialization)\\Conferred with high distinction}{2008}


\vspace{-1em}
\section*{Professional Experience}
\vspace{-0.75em}
\setlist[itemize]{leftmargin=0.2in}

\dancol{\textbf{Assistant Professor},
University of California Los Angeles, \\Departments of Computational Medicine and Neurology, UCLA Brain Mapping Center\\
\vspace{-0.3in}
\begin{itemize}[noitemsep]
\item 

Developing brain image analysis algorithms for disease modeling, diagnosis, and prognosis, including analysis of electronic health records.

\end{itemize}
}{2020-\phantom{20}}


\dancol{\textbf{Assistant Research Professor},
Johns Hopkins University, \\Department of Biomedical Engineering\\
\vspace{-0.3in}
\begin{itemize}[noitemsep]
\item 

Developed brain analysis and mapping techniques that cross spatial scales, for multiple species and image modalities, including serial sectioning, brain clearing (CLARITY/iDISCO), and synaptic labeling with sub micron resolution.  


\end{itemize}
}{2019-20}


\dancol{\textbf{Kavli Distinguished Postdoctoral Fellow} mentored by Dr. Juan Troncoso (Department of Pathology) and Dr. Michael Miller (Department of Biomedical Engineering) \\
Johns Hopkins University, Kavli Neuroscience Discovery Institute\\
\vspace{-0.3in}
\begin{itemize}[noitemsep]
\item 
%Bridging the gap between longitudinal clinical imaging and autopsy findings in Alzheimer's disease.

%Developed multi modality image mapping and analysis techniques, connecting the micron to the mm scale.

Developed computational tools to connect longitudinal clinical human MR at millimeter scale to histopatholgy at micron scale, a crucial gap in our understanding of Alzheimer's progression and diagnosis. 


%Brain mapping crossing scales in multiple species connecting behavioral scale and MR scales in human to animal species in cleared brains and .... at micron scales.


\end{itemize}
}{2017-19}

\dancol{\textbf{Graduate Research Assistant} mentored by Dr. Michael Miller\\
Johns Hopkins University School of Medicine, Center for Imaging Science\\
\vspace{-0.3in}
\begin{itemize}[noitemsep]
\item Designed parsimonious representations of anatomical variability in a Bayesian setting, and robust algorithms for computing structural imaging biomarkers of neurodegenerative disease.
\end{itemize}}
{2009-17}

\dancol{\textbf{Research Assistant} mentored by Dr. Jeffrey Siewerdsen\\
Image Guided Therapy Group, Department of Biophysics and Bioimaging, Ontario Cancer Institute, University Health Network\\
\vspace{-0.3in}
\begin{itemize}[noitemsep]
\item 
%Theoretically and experimentally investigating the image quality of 3D cone-beam computed tomography and tomosynthesis within a task-specific framework.
Created an experimentally verified theoretical model for image quality in 3D cone-beam computed tomography and tomosynthesis, using detector physics and human psychophysics.
\end{itemize}
}{2007-09}

\dancol{\textbf{Summer Student} mentored by Dr. Jeffrey Siewerdsen\\
Department of Medical Biophysics, University of Toronto\\
\vspace{-0.3in}
\begin{itemize}[noitemsep]
\item 
%Examining limits of imaging performance in low x-ray dose cone-beam computed tomography and dual energy radiography, designing tools for evaluation of image quality using human observer performance.
Established low dose limits for adequate imaging performance in cone-beam computed tomography and dual energy radiography. Designed computational tools for evaluating image quality using human observer performance.
\end{itemize}
}{2005-06}

\comment{
\section*{Funding Awarded}

\dancol{\textbf{XSEDE Research Grant for Computational Resources} (PI, 3,607,906
CPU hours, 50,000 GPU hours, \$33,767.76 estimated value).  ``Computational Anatomy Gateway''.}{2019}



\dancol{\textbf{XSEDE Research Grant for Computational Resources} (PI, 3,856,160 CPU hours, 55,000 GPU hours, \$43,009.03 estimated value).  ``Computational Anatomy Gateway''.}{2018}

\dancol{\textbf{Kavli Neuroscience Discovery Institute Distinguished Postdoctoral Fellowship} (\$60,000/yr), ``MR Micro Imaging and Three Dimensional Histology: Integrating Neuroimaging Data Across Scales.'' }{2017-18}

\dancol{\textbf{XSEDE Research Grant for Computational Resources} (PI, 2,580,000 CPU hours, \$55,918.00 estimated value).  ``Computational Anatomy Gateway''.}{2017}

\dancol{\textbf{XSEDE Research Grant for Computational Resources} (Co-PI, 2,823,800 CPU hours, \$168,083.43 estimated value).  ``Computational Anatomy Gateway''.}{2016}

\dancol{\textbf{XSEDE Research Grant for Computational Resources} (Co-PI, 1,363,260 CPU hours, \$81,537.90 estimated value).  ``Computational Anatomy Gateway''}{2015}

\dancol{\textbf{NSERC Doctoral Postgraduate Scholarship} (\$42,000 CAD/yr). Research funding from Natural Sciences and Engineering Research Council (NSERC) of Canada}{2012-14}

\dancol{\textbf{Julie Payette NSERC Research Scholarship} (\$25,000 CAD). Masters-level research funding with distinction for scoring in the top 24 applications to NSERC of Canada}{2010}


\lhead{\name} % not until the second page

\section*{Academic Honors}

\dancol{\textbf{Shape Special Interest Group Best Paper award (\$300)}.  Judged best paper and presentation in Mathematical Foundations of Computational Anatomy at MICCAI conference}{2019}

\dancol{\textbf{XSEDE Best Lightning Talk}.  Judged best presentation in Lightning Talk session, a format for presenting top-scoring papers in each track conference-wide}{2014}

\dancol{\textbf{Farrington Daniels Award} (\$500). Best paper on radiation dosimetry published in \textit{Medical Physics} in 2012}{2013}

\dancol{\textbf{Johns Hopkins Imaging Initiative First Place Poster Award} (\$100).  Judged best poster at an institutional imaging conference}{2013}

\dancol{\textbf{Certificate Award of Excellence for Trainee Presentation} (\$250 CAD).  Judged best student talk at MITACS workshop of Mathematics of Brain Imaging}{2012}

\dancol{\textbf{The Prince Phillip Silver Medal}.  For achieving the second-highest GPA in the sciences in my graduating class}{2008}

\dancol{\textbf{The Scott Memorial Scholarship} (\$300 CAD). Academic achievement upon graduating}{2008}

\dancol{\textbf{Summer Undergraduate Studentship},  (\$1125 CAD) from University of Toronto's Life Sciences Awards Committee. Exceptional performance as a summer student in medical biophysics }{2006}

\dancol{\textbf{University of Toronto Scholar} (\$500 CAD). Academic achievement}{2006}

\dancol{\textbf{Professor William Kingston and Dr John Kingston Scholarship} (\$1000 CAD). Academic achievement}{2006}

\dancol{\textbf{Harold E. Johns Summer Studentship in Medical Physics Award} (\$4200 CAD). Exceptional performance as a summer student in medical physics}{2005}

\dancol{\textbf{McCutcheon Award} (\$1,000 CAD).  Academic achievement}{2005}

\dancol{Induction into \textbf{Golden Key International Honor Society} (\$90 CAD) %recognizing outstanding academic achievement and connecting high-achieving individuals locally, regionally, and globally
}{2005}

\dancol{\textbf{Mrs. F. N. G. Starr Scholarship} (\$1,000 CAD). Academic achievement}{2004}
}



\setlist[enumerate]{leftmargin=0.2in}

\section*{Recent Publications}


\begin{enumerate}

\item
\fullcite{tward2020diffeomorphic}
\item
\fullcite{charles2020toward}
%\item
%\fullcite{miller2020coarse}
%\item
%\fullcite{lee2020infinitesimal}

\item 
\fullcite{tward2019estimating}

\item
\fullcite{kulason2018corticala}

\item 
\fullcite{lee2018variational}

\item 
\fullcite{vogelstein2018community}

\item
\fullcite{miller2018computational}

\item 
\fullcite{tward2017entorhinal}
\item 
\fullcite{tward2017complexity}
\item 
\fullcite{tward2017parametric}

\item 
\fullcite{faria2016linking}

\comment{
\item 
\fullcite{segars2015development}
\item 
\fullcite{miller2015network}


\item 
\fullcite{mahon2015morphometry}
\item 
\fullcite{miller2015amygdalar}

\item 
\fullcite{norris2014set}

\item 
\fullcite{tward2013robust}
\item 
\fullcite{segars2013population}

\item 
\fullcite{pineda2012beyond}
\item 
\fullcite{li2012effects}

\item 
\fullcite{tward2011patient}
\item 
\fullcite{gang2011analysis}

\item 
\fullcite{gang2010anatomical}

\item \fullcite{tward2009noise}

\item \fullcite{tward2008cascaded}

\item \fullcite{tward2007soft}
\item \fullcite{williams2007optimal}
}
\end{enumerate}


\end{document}